% Build with
%   pdflatex -shell-escape fraction_to_float.tex
% The -shell-escape argument is necessary for the syntax highlighting of
% the Python code, which calls out to pygments.

\documentclass[a4paper]{article}

\usepackage{minted}
\usepackage{amsthm}
\usepackage{mathtools}
\usepackage{sourcecodepro}
\usepackage[T1]{fontenc}

\DeclarePairedDelimiter\abs{\lvert}{\rvert}

\theoremstyle{plain}
\newtheorem{theorem}{Theorem}


\title{Injectivity of fraction to float conversion}
\author{Mark Dickinson}

\begin{document}

\maketitle

It can occasionally be convenient to store fractions as floating-point numbers.
Any mapping from the (infinite) set of all possible fractions to a finite set
of limited-precision floats must necessarily be lossy: there will be examples
of different fractions which map to the same float. But with a small enough
bound on the numerator and denominator of the fractions considered, the
function which maps each such bounded fraction to the nearest exactly
representable float becomes injective, making it possibly to unambiguously
recover the fraction from its floating-point representation. The best possible
bound will depend on the target floating-point format.

This note establishes that for IEEE 754 binary64 floating-point (often referred
to as ``double precision''), $67114657$ is such a bound. This bound is best
possible, since the distinct fractions $67114658 / 67114657$ and $67114657 /
67114656$ round to the same IEEE 754 binary64 float under round-to-nearest.

\begin{theorem}
    Suppose that $a/b$ and $c/d$ are rational numbers, written in lowest terms
    (with $b$ and $d$ positive), that $\max(\abs{a}, b, \abs{c}, d) \le
    67114657$, and that $a/b$ and $c/d$ become equal when rounded to the
    nearest finite IEEE binary64 floating-point number. Then $a/b = c/d$.
\end{theorem}

\begin{proof}
    It's straightforward to check that when we round a fraction with numerator
    and denominator bounded by~$67114657$ to a float, no underflow or overflow
    occurs, and that under the same bounds positive fractions round to positive
    floats and negative fractions to negative floats. So if $a/b$ and $c/d$
    round to the same float then both are positive, both are negative, or both
    are zero, and in the last case we're done. Negating both fractions if
    necessary, without loss of generality we can assume going forward that all
    of $a$, $b$, $c$ and $d$ are positive.

    Now we separate into two cases. The first case we consider is the case in
    which $a/b$ and $c/d$ belong to the same closed binade---that is, there's
    an integer~$e$ such that both $a/b$ and $c/d$ lie in the interval $[2^e,
    2^{e+1}]$.
    Since consecutive IEEE 754 binary64 floats in that interval have
    difference~$2^{e-52}$, for $a/b$ and $c/d$ to round to the same float we
    must have
    $\abs{a/b - c/d} \le 2^{e-52}$, from which
    \begin{equation}
        2^{52-e}\abs{ad - bc} \le bd.\label{ineq:1}
    \end{equation}
    From this inequality combined with $2^e \le a/b$ and $2^e \le c/d$, we
    have:
    \begin{equation}
        2^{52+e}\abs{ad - bc} \le 2^{2e}bd \le ac  .\label{ineq:2}
    \end{equation}
    Using (\ref{ineq:1}) if $e \le 0$ and (\ref{ineq:2}) if $e \ge 0$, along
    with the bound on~$\max(a, b, c, d)$, we have
    \begin{equation}
        2^{52 + \abs{e}}\abs{ad - bc} \le 67114657^2 < 2^{53}.\label{ineq:3}
    \end{equation}
    hence
    \begin{equation}
        2^{\abs{e}}\abs{ad - bc} < 2.\label{ineq:4}
    \end{equation}
    It follows that either $\abs{ad - bc} = 0$ (in which case $a/b = c/d$ as
    required), or $\abs{ad - bc} = 1$ and $e = 0$. In this case the inequality
    (\ref{ineq:1}) gives $2^{52} \le bd$, and the inequalities $2^e \le a/b$
    and $2^e \le c/d$ give $b \le a$ and $c \le d$. At this point we look for
    a contradiction.

    Swapping $a/b$ and $c/d$ if necessary, we can assume that $d \le b$, so
    from $2^{52} \le bd$ we have $2^{26} \le b \le a$. We can further deduce
    that $d < b$ (if $d = b$ then $1 = |ad - bc|$ is divisible by $b$, which
    is impossible), and that $0 < c < a$.

    Now it's possible to do an exhaustive search over all pairs $(a, b)$ of
    integers satisfying
    \[2^{26} \le b \le a \le 67114657.\] There are exactly $16788115$ such
    pairs (reducing to $10204542$ after we discard those with $\gcd(a, b) \ne
    1$), making this search computationally very feasible. For each pair $(a,
    b)$ we can use the extended Euclidean algorithm to efficiently find all
    possible solutions in integers $c$ and $d$ to $\abs{ad - bc} = 1$ with $0 <
    d \le b$ (there are at most two), and check that $a/b$ and $c/d$ do not
    round to the same float, giving us our contradiction.

    There remains the second case, where there is no integer $e$ such that
    both $a/b$ and $c/d$ lie in $[2^e, 2^{e+1}]$. The only way for this to
    be possible is if there's a power of two separating $a/b$ and $c/d$; that
    is, without loss of generality (swapping $a/b$ and $c/d$ if necessary)
    there's an $e$ satisfying
    \[a/b < 2^e < c/d.\]
    From the bounds on $a$, $b$, $c$ and~$d$, we must have
    $-26 \le e \le 26$. But now $2^e$ can be expressed as a fraction with
    numerator and denominator not exceeding $67114657$, and since $a/b$ and
    $c/d$ round to the same float, $2^e$ (being squeezed between $a/b$ and
    $c/d$) must round to that same float. So we can apply the case 1 proof to
    $a/b$ and $2^e$ to deduce that $a/b = 2^e$, and again to $2^e$ and $c/d$ to
    deduce that $2^e = c/d$, hence that $a/b = c/d$, as required.
\end{proof}

Listing \ref{lst:exhaustive} shows Python code for the exhaustive search;
rather than stopping once it reaches $a = 67114657$, it keeps going to print
all solutions found to $a/b = c/d$. It produces the first result in under 30
seconds on my laptop.

\begin{listing}
\inputminted{python}{fraction_to_float.py}
\caption{Exhaustive search}
\label{lst:exhaustive}
\end{listing}

\end{document}
